%-------------------------------------------------------------------------------
%	SECTION TITLE
%-------------------------------------------------------------------------------
\cvsection{Latest Projects}
\vspace*{0.02in}
%-------------------------------------------------------------------------------
%	CONTENT
%-------------------------------------------------------------------------------
\begin{cventries}
%---------------------------------------------------------
	\cventry
    	{Academic project} % Affiliation/role
    	{Intelligent Parking Management using camera at lamp-post angle}
    	{The University of Texas at Arlington} % Location
    	{Feb. 2017 - May 2017} % Date(s)
    	{
      		\begin{cvitems} % Description(s) of experience/contributions/knowledge
        		\item {Detects empty and filled parking spaces  using a combination of HAAR classifier, laplacian operator, background subtraction and motion tracking} 
        		\item {Implemented on Real-time parking lot videos from one of the university's parking using OpenCV, Python, Numpy, Scipy, and YAML. Explanation can be found \href{https://youtu.be/y1M5dNkvCJc}{here}. The video has 7300+  views on YouTube. For code, please visit \href{https://github.com/ankit1khare/Automatic-Parking-Management}{this  repository}}
        		\vspace*{0.01in}
        	\end{cvitems}
        }		
	\cventry    		
		{Individual project}
		{Deep Neural Network based parking management system}        			{The University of Texas at Arlington}
		{Nov. 2018 - Jan. 2018}
		{
			\begin{cvitems}
        		\item {Improved the efficiency of the vehicle classifier by switching from HAAR to Resnet101 trained using Fastai library} 
        		\item {Implemented using Fastai, Pytorch and Python. Code can be found \href{https://github.com/ankit1khare/Deep-Neural-Network-based-parking-system}{here}}  
        		\vspace*{0.09in}
      		\end{cvitems}
    	}
    \cventry    		
		{Individual project}
		{Real-Time parking spot vacancy alert}        			
		{The University of Texas at Arlington}
		{Dec. 2018 - Present}
		{
			\begin{cvitems}
        		\item {Implemented YOLO V3 to compare and contrast its performance with MASK-RCNN. Code can be seen   \href{https://github.com/ankit1khare/Smart-Park-with-YOLO-V3}{here}} 
        		\item {Made a prototype to alert a person by sending text on his/her cell phone whenever any parking spot is vacant on nearby street. Idea can be implemented on real-time systems. Check \href{https://github.com/ankit1khare/Easy_street_parking_with_MASK-RCNN}{here} for further details}
        		\vspace*{0.09in}  
      		\end{cvitems}
    	}
	\cventry
    	{Individual project} % Affiliation/role
    	{TimeIT} 
    	{Intelligent Interval Setter}
    	{Jan. 2019 - Present} % Date(s)
    	{
      		\begin{cvitems} % Description(s) of experience/contributions/knowledge
        		\item {A cross-platform app to set time intervals for alarm using voice input. Will use Natural Language Processing (NLP) to paraphrase and set duration/number of intervals. Suitable for assistance during exercise, yoga, taking medications etc. For further details visit the \href{https://github.com/ankit1khare/Seq-2-Seq-on-custom-data}{repository}}
        		\vspace*{0.06in}
      		\end{cvitems}
    	} 
%    \cventry
%	    {Individual project} % Affiliation/role
%    	{Drawing Colorization} 
%    	{Draw on-screen and get it colorized}
%    	{Jan. 2019 - PRESENT} % Date(s)
%    	{
%      		\begin{cvitems} % Description(s) of experience/contributions/knowledge
%        		\item {Aim is to detects finger movements to draw a quick representation of a simple object like duck, airplane, person etc. and then use neural networks to colorize it in anime style}
%      		\end{cvitems}
%    	}   
%---------------------------------------------------------
\end{cventries}